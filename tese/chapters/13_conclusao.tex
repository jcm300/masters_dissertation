Esta dissertação tem vários objetivos distintos mas todos com a motivação de continuar o desenvolvimento da plataforma \acrshort{clav} cumprindo alguns dos requisitos em falta na plataforma.

A documentação em \textit{OpenAPI} (nova versão do \textit{Swagger}) da \acrshort{api} de dados da \acrshort{clav} desenvolvida está presente em \url{https://clav-api.dglab.gov.pt/v2/docs/} e facilita assim o desenvolvimento da plataforma \acrshort{clav} bem como quem desenvolve sistemas arquivísticos que irão usar esta \acrshort{api} visto a documentação possuir uma descrição em cada rota, os parâmetros que se podem usar bem como exemplos e as estruturas dos \textit{bodies} e das respostas.

A exportação na \acrshort{api} de dados da \acrshort{clav} dos principais tipos de dados em \acrshort{xml}, \acrshort{csv} e \acrshort{rdf} foi também alcançada, tendo ainda sido desenvolvido adicionalmente uma interface de exportação que permite escolher o \textit{enconding} de saída. Permite assim que os utilizadores possam, por exemplo visualizar/transformar os dados em \textit{Excel} ou ferramentas similares para além de permitir que aplicações possam usar outros formatos para tratar estes dados, por exemplo, ao exportar em \acrshort{xml}.

Por outro lado, foi protegida a \acrshort{api} de dados com múltiplos níveis de acesso, garantindo assim que os dados presentes na \acrshort{api} da \acrshort{clav} não são alterados indevidamente por utilizadores mal intencionados. Assim, apenas pessoal autorizado (de acordo com o nível de utilizador) pode fazer alterações.

Além disso, é agora também possível entrar na plataforma \acrshort{clav} através de \acrfull{cmd}, um dos mecanismos de autenticação disponíveis no Autenticação.gov. É dada assim mais uma hipótese de autenticação aos utilizadores para além das já implementadas através de \acrshort{cc} e de \textit{email}+\textit{password}.

Através do \textit{Kong} foi desenvolvida uma \textit{\acrshort{api} Gateway} separando da \acrshort{api} de dados toda a proteção, colocando esta num serviço independente. O uso do \textit{Kong} permitirá futuramente se assim se pretender modularizar ainda mais a \acrshort{api} de dados em microserviços. Da perspetiva dos utilizadores a proteção da \acrshort{api} de dados manteve-se igual, mas permitiu simplificar a complexidade do código da \acrshort{api} de dados.

Quanto ao objetivo de integração da \acrshort{clav} no \acrshort{iap} este é conseguido com o uso do Autenticação.gov para autenticar os utilizadores a quem é requisitado a obtenção de dados pessoais do utilizador, como o \acrshort{nic} e o nome completo para autenticar o utilizador na \acrshort{clav}.

São assim atingidos todos os objetivos da dissertação. Adicionalmente realizou-se a migração de \acrshort{http} para \acrshort{https} melhorando a segurança de utilização da plataforma para além da exigência desta para se poder utilizar a versão de produção do Autenticação.gov. Além disso, apesar de não referidas neste documento, já ele extenso, foram adicionados mecanismos de \textit{cache} na \acrshort{api} de dados, otimizado o cálculo do fecho transitivo da \acrshort{lc}, desenvolvido a importação de uma \acrshort{ts} a partir de um ficheiro, desenvolvida uma interface de pesquisa avançada para a \acrshort{lc} e uma interface para o registo de utilizadores para uma entidade, adicionada a filtragem das classes da \acrshort{lc} de acordo com o nível do utilizador e com o \textit{status} de cada classe e foi adicionado a validação de parâmetros na \acrshort{api} de dados com recurso ao \texttt{express-validator} garantindo uma maior resiliência da \acrshort{api} e devolvendo um maior \textit{feedback} aos utilizadores indicando que parâmetros estão errados e porquê. Foi também adicionada a capacidade de guardar os \textit{logs} dos pedidos efetuados à \acrshort{api}, sua consulta e sua gestão e vários mecanismos de gestão da plataforma, como alteração de parâmetros, exportar e apagar coleções bem como a gestão da cache.

Quanto às questões de investigação introduzidas no inicio desta dissertação, a produção da documentação da \acrshort{api} de dados é realizada através de múltiplos ficheiros por forma a modularizar a documentação. Estes ficheiros estão organizados estruturalmente de acordo com o seu papel. Por forma, a gerar o documento final com a documentação em \textit{OpenAPI} é usado o \texttt{yaml-include}. O documento final é por fim usado pelo \textit{SwaggerUI}, através do \texttt{swagger-ui-express}, para produzir uma documentação dinâmica.

Relativamente à \textit{\acrshort{api} gateway} esta permite várias vantagens numa arquitetura em microserviços, desde oferecer uma vista única (único ponto de entrada) da \acrshort{api}, permitir esconder a arquitetura interna bem como simplificar a autenticação e autorização da \acrshort{api} já que há só uma entrada a proteger (a superfície de ataque é menor se compararmos com uma arquitetura em microserviços em que cada microserviço tem a sua própria autenticação e autorização).

Por fim, a autonomização do serviço de autenticação traz como benefícios:
\begin{itemize}
    \item os pedidos sem autenticação não chegam agora à \acrshort{api} de dados reduzindo a carga sobre esta;
    \item o serviço de autenticação, sendo este um microserviço, pode ser colocado numa máquina à parte;
    \item simplificação do código da \acrshort{api} de dados, melhorando a facilidade de manutenção deste;
    \item maior facilidade na manutenção da autenticação da \acrshort{api} de dados, por exemplo na alteração das permissões de acesso de determinada rota.
\end{itemize}

\section{Trabalho Futuro}

Foram alcançados todos os objetivos desta dissertação, contudo certas tarefas requerem uma manutenção constante. Uma delas é a documentação em \textit{OpenAPI} que necessita de ser atualizada sempre que são adicionadas novas rotas à \acrshort{api} ou quando alguma das rotas presentes mude, seja na resposta, nos parâmetros, na funcionalidade ou ainda caso esta passe a \textit{deprecated}. Outra tarefa que requer manutenção é a exportação para \acrshort{csv}. Se se pretender que mais alguma rota possua exportação para \acrshort{csv} ou caso alguma das respostas das rotas que possuem exportação para \acrshort{csv} sejam alteradas é necessário adicionar ou alterar, respetivamente, no conversor de \acrshort{json} para \acrshort{csv} como deve ser feita a transformação dos objetos \acrshort{json} de saída da rota. Há ainda outra tarefa de manutenção e esta deve também ser realizada sempre que é adicionada uma nova rota. Esta tarefa consiste em adicionar ao serviço de \textit{Auth}, na versão com \textit{Kong}, quem pode aceder a essa rota, ao adicionar as permissões dessa rota no dicionário de permissões das rotas presente nesse serviço.

Já quanto à continuação do trabalho desenvolvido por esta dissertação destaca-se a \acrshort{api} de dados com \textit{Kong}. Apesar do que foi desenvolvido estar num estado passível de ser usado em produção, há varias melhorias que podem ser desenvolvidas desde replicar a \acrshort{api} de dados colocando o \textit{Kong} a servir de \textit{load balacing} até colocar \textit{rate limiting} especificamente para cada rota com o objetivo principal de melhorar a performance e a tolerância a falhas da \acrshort{api} de dados.
