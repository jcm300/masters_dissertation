A \acrlong{ap} portuguesa tem desmaterializado processos e tem promovido a adoção de sistemas de gestão documental eletrónica bem como a digitalização de documentos destinados a serem arquivados. Estas medidas pretendem atingir a otimização de processos, a modernização de procedimentos administrativos e a redução de papel.

Com o propósito de atingir estes objetivos e simplificar a gestão documental na \acrlong{ap}, a \acrfull{clav} nasce como uma das medidas. A \acrshort{clav} tem como finalidade a classificação e a avaliação da informação pública por forma a auxiliar os sistemas de informação das entidades públicas alertando-as quando determinado documento deve ser arquivado ou eliminado. Para tal esta possui um referencial comum, a \acrlong{lc}, com as funções e processos de negócio das entidades públicas associadas a um catálogo de legislação e de organismos.

Esta dissertação tem como principais objetivos a proteção da \acrshort{api} da \acrshort{clav}, a documentação desta \acrshort{api}, a adição de formatos exportação a esta bem como a continuação da integração do Autenticação.gov na \acrshort{clav} ao adicionar a possibilidade de autenticação com a \acrlong{cmd}. Tem ainda como objetivo a criação de uma \acrshort{api} Gateway na \acrshort{clav}.

\vspace{1cm}

\textit{\textbf{Keywords:}} API Gateway, Autenticação, Autenticação.gov, CLAV, Swagger  
