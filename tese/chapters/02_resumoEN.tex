The portuguese public administration has dematerialized processes and promoted the adoption of electronic document management systems as well as the scanning of documents intended to be archived. This measures aim to optimize and modernize administrative procedures and reduce paper usage.

In order to achieve these objectives and simplify the document management in public administration, \acrshort{clav} was born as one of the measures. \acrshort{clav}'s main purpose is the classification and evaluation of the public information in order to help the information systems of public entities, alerting them when certain documents must be filed or deleted. To this end, a common reference, called the \textit{consolidated list} (\acrlong{lc}), is used, with the business functions and processes of public entities associated with a catalogue of legislation and entities.

In the last two years, \acrshort{clav} have been develop in computing department of \acrshort{um} in strict collaboration with the area investigation team at \acrlong{dglab}.

At start date of this work, \acrshort{clav} was constituted by two database servers that had as interlocutor the data \acrshort{api} server of \acrshort{clav}. Was with this data \acrshort{api} server that all exterior interaction passed: access from third party applications and access from client interface developed for \acrshort{clav}.

In this dissertation, the big goal was make evolve the application architecture giving answer to a series of requirements and trying to simplify to maximum the process of it future maintenance.

In this sense, a protection service for data \acrshort{api} of \acrshort{clav} was specified and developed, the data \acrshort{api} documentation was specified, the exportation formats were defined and the \acrshort{api} exporters were developed in order to allow a bigger data interoperability, the authentication with \acrlong{cmd} using the Autenticação.gov was developed, the necessary mechanisms of \acrshort{http} to \acrshort{https} migration were created and, lastly, an \acrshort{api} Gateway on \acrshort{clav} was added in order to simplify the operation and management of the platform.

All these developments are in production and can be observed accessing to the official web page of \acrshort{clav}: \url{https://clav.dglab.gov.pt}

\vspace{1cm}

\textbf{Keywords:} API Gateway, Autenticação.gov, Authentication, CLAV, Swagger
