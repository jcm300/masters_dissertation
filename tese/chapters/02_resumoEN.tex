The portuguese public administration has dematerialized processes and promoted the adoption of electronic document management systems as well as the scanning of documents intended to be archived. This measures aim to optimize and modernize administrative procedures and reduce paper usage.

In order to achieve these objectives and simplify the document management in public administration, \acrshort{clav} was born as one of the measures. \acrshort{clav}'s main purpose is the classification and evaluation of the public information in order to help the information systems of public entities, alerting them when certain documents must be filed or deleted. To this end, a common reference, called the \textit{consolidated list} (\acrlong{lc}), is used, with the business functions and processes of public entities associated with a catalog of legislation and entities.

This dissertation has as main objectives the \acrshort{clav} \acrshort{api} protection, the documentation for this \acrshort{api}, adding export formats as well as the continued integration of Autenticação.gov into \acrshort{clav} by adding the possibility of authentication with the \textit{digital mobile key} (\acrlong{cmd}). It also aims to create an \acrshort{api} Gateway at \acrshort{clav}.

\vspace{1cm}

\textbf{Keywords:} CLAV, Swagger, Autenticação.gov, API Gateway, Authentication
