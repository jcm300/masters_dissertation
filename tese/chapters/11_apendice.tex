\chapter{Algoritmos}

\section{Conversor de \acrshort{json} para \acrshort{xml}}\label{exem:convXML}
\begin{lstlisting}[language=pseudocode, caption=Algoritmo de conversão de \acrshort{json} para \acrshort{xml}]
sizeTab = 4

function protectForXml(string)
    string = replace '<' by '&lt;' in string
    string = replace '>' by '&gt;' in string
    string = replace '&' by '&amp;' in string
    string = replace "'" by '&apos;' in string
    string = replace '"' by '&quot;' in string
    return string

function protectKey(string)
    string = replace '<' by '' in string
    string = replace '>' by '' in string
    string = replace '&' by '_' in string
    string = replace "'" by '' in string
    string = replace '"' by '' in string
    string = replace '\s+' by '_' in string
    return string

function json2xmlArray(array, nTabs)
    xml = ''
    len = length(array)

    for i=0; i < len; i++
        type = type of array[i]
        xml += repeat(' ', nTabs * sizeTab) + '<item index="' + i + '" type="' + type + '">' 

        if type == 'object' or type == 'string' or type == 'boolean' or type == 'number' then
            xml += json2xmlRec(array[i], nTabs + 1)

        if type == 'object' then
            xml += repeat(' ', nTabs * sizeTab)

        xml += '</item>\n'

    return xml

function json2xmlRec(json, nTabs)
    xml = ''
    type = type of json
    
    if type == 'object' then
        xml = '\n'

        if json is an Array then
            xml = json2xmlArray(json, nTabs)
        else
            for key in json
                aux = ''
                type = type of json[key]

                if type === 'object' then
                    if json[key] is an Array then
                        aux = '\n' + json2xmlArray(json[key], nTabs + 1)
                        type = 'array'
                    else
                        aux += json2xmlRec(json[key], nTabs + 1)
                    
                    aux += repeat(' ', nTabs * sizeTab)
                else if type == 'string' then
                    aux = protectForXml(json[key])
                else if type == 'boolean' or type == 'number' then
                    aux = json[key]

                if json[key] !== undefined then
                    xml += repeat(' ', nTabs * sizeTab)
                    xml += '<' + protectKey(key) + ' type="' + type + '">'
                    xml += aux + '</' + protectKey(key) + '>\n'
    else if type == 'string' then
        xml = protectForXml(json)
    else if type == 'boolean' or type == 'number' then
        xml = json

    return xml

function json2xml(json)
    xml = '<?xml version="1.0" encoding="utf-8"?>\n'
    xml += '<root>'
    xml += json2xmlRec(json, 1)
    xml += '</root>'
    return xml
\end{lstlisting}

\section{Conversor de \acrshort{json} para \acrshort{csv}}\label{exem:convCSV}
\begin{lstlisting}[language=pseudocode, caption=Algoritmo de conversão de \acrshort{json} para \acrshort{csv}]
separator = ';'
types = ["classes", "classe", "pesquisaClasses", "preselecionadoClasses", "esqueletoClasses", "entidades", "entidade", "tipologias", "tipologia",  "legislacoes", "legislacao"]
convert_to = {
    "classe": {
        "codigo": ["Código", v => v],
        "titulo": ["Título", v => v],
        "descricao": ["Descrição", v => v],
        "notasAp": ["Notas de aplicação", map_value("nota")],
        "exemplosNotasAp": ["Exemplos de NA", map_value("exemplo")],
        "notasEx": ["Notas de exclusão", map_value("nota")],
        "termosInd": ["Termos Indice", map_value("termo")],
        "tipoProc": ["Tipo de processo", v => v],
        "procTrans": ["Processo transversal (S/N)", v => v],
        "donos": ["Donos do processo", map_value("sigla")],
        "participantes_sigla": ["Participante no processo", map_value("sigla")],
        "participantes_participLabel": ["Tipo de intervenção do participante", map_value("participLabel")],
        "processosRelacionados_codigo": ["Código do processo relacionado", map_value("codigo")],
        "processosRelacionados_titulo": ["Título do processo relacionado", map_value("titulo")],
        "processosRelacionados_idRel": ["Tipo de relação entre processos", map_value("idRel")],
        "legislacao_idLeg": ["Diplomas jurídico-administrativos REF Ids", map_value("idLeg")],
        "legislacao_titulos": ["Diplomas jurídico-administrativos REF Títulos", leg_titulos],
        "pca": ["", v => convertOne(v, "pca")],
        "df": ["", v => convertOne(v, "df")],
        "filhos": [null, filhos("classe")]
    },
    "preselecionadoClasse": {
        "codigo": ["Código", v => v],
        "titulo": ["Título", v => v],
        "descricao": ["Descrição", v => v],
        "dono": ["Dono", v => v],
        "participante": ["Participante", v => v],
        "pca": ["PCA", v => v],
        "formaContagem": ["Forma de contagem do PCA", v => v],
        "df": ["DF", v => v]
    },
    "esqueletoClasse": {
        "codigo": ["Código", v => v],
        "titulo": ["Título", v => v],
        "descricao": ["Descrição", v => v],
        "dono": ["Dono", v => v],
        "participante": ["Participante", v => v],
        "pca": ["PCA", v => v],
        "df": ["DF", v => v]
    },
    "pesquisaClasse": {
        "id": ["Código", v => v],
        "titulo": ["Título", v => v],
        "tp": ["Tipo de processo", v => v],
        "pt": ["Processo transversal (S/N)", v => v],
        "na": ["Notas de aplicação", v => v],
        "exemploNa": ["Exemplos de NA", v => v],
        "ne": ["Notas de exclusão", v => v],
        "ti": ["Termos Indice", v => v],
        "pca": ["Prazo de conservação administrativa", v => v],
        "fc_pca": ["Forma de contagem do PCA", v => v],
        "sfc_pca": ["Sub Forma de contagem do PCA", v => v],
        "crit_pca": ["Critério PCA", join],
        "df": ["Destino final", v => v],
        "crit_df": ["Critério DF", join],
        "donos": ["Donos do processo", join],
        "participantes": ["Participantes do processo", join],
        "filhos": [null, filhos("pesquisaClasse")]
    },
    "entidade": {
        "sigla": ["Sigla", v => v],
        "designacao": ["Designação", v => v],
        "estado": ["Estado", v => v],
        "sioe": ["ID SIOE", v => v],
        "internacional": ["Internacional", internacional],
        "dono": ["Dono no processo", map_value("codigo")],
        "participante_codigo": ["Participante no processo", map_value("codigo")],
        "participante_tipoPar": ["Tipo de intervenção no processo", map_value("tipoPar")],
        "tipologias": ["Tipologias da entidade", map_value("sigla")]
    },
    "tipologia": {
        "sigla": ["Sigla", v => v],
        "designacao": ["Designação", v => v],
        "estado": ["Estado", v => v],
        "entidades": ["Entidades da tipologia", map_value("sigla")],
        "dono": ["Dono no processo", map_value("codigo")],
        "participante_codigo": ["Participante no processo", map_value("codigo")],
        "participante_tipoPar": ["Tipo de intervenção no processo", map_value("tipoPar")]
    },
    "legislacao": {
        "tipo": ["Tipo", v => v],
        "numero": ["Número", v => v],
        "data": ["Data", v => v],
        "sumario": ["Sumário", v => v],
        "fonte": ["Fonte", v => v],
        "link": ["Link", v => v],
        "entidades": ["Entidades", map_value("sigla")],
        "regula": ["Regula processo", map_value("codigo")]
    },
    "pca": {
        "valores": ["Prazo de conservação administrativa", v => v],
        "notas": ["Nota ao PCA", v => v],
        "formaContagem": ["Forma de contagem do PCA", v => v],
        "subFormaContagem": ["Sub Forma de contagem do PCA", v => v],
        "justificacao_criterio": ["Critério PCA", map_value("tipoId")],
        "justificacao_refs": ["ProcRefs/LegRefs PCA", refs]
    },
    "df": {
        "valor": ["Destino final", v => v],
        "notas": ["Notas ao DF", v => v],
        "justificacao_criterio": ["Critério DF", map_value("tipoId")],
        "justificacao_refs": ["ProcRefs/LegRefs DF", refs]
    }
}

internacional(value)
    if value == "Sim" then
        return "Sim"
    else
        return "Não"

join(array)
    return join_with(array, '#\n')

map_value(key)
    return function(value)
        return join(map(value, p => p[key]))

leg_titulos(value)
    return join(map(value, l => l.tipo + ' ' + l.numero))

refs(value)
    procs_legs = []

    for just in value
        if len(just.processos) > 0 then
            procs_legs.push('(' + join(map(just.processos, p => p.procId)) + ')')
        else if len(just.legislacao) > 0 then
            procs_legs.push('(' + join(map(just.legislacao, l => l.legId)) + ')')
        else
            procs_legs.push('()')

    return join(procs_legs)

filhos(value)
    csvLines = []

    for classe in value
        aux = convertOne(classe, "classe")
        delete aux[0]
        csvLines = concat(csvLines, aux)

    return csvLines

protect(string)
    if string != null then
        if type of string == 'string' then
            string = replace '"' by '""' in string
        return '"' + string + '"'
    else
        return '""'

joinLines(csvLines)
    len = len(csvLines)

    for i = 0; i<len; i++
        csvLines[i] = join_with(csvLines[i], separator)

    return join_with(csvLines, '\n')

convertOne(json, type)
    csvLines = [[],[]]

    for key in convert_to[type]
        k = (split key by '_')[0]
        header = convert_to[type][key][0]

        if k in json then
            f = convert_to[type][key][1]
            value = f(json[k])

            if header == null then
                csvLines = concat(csvLines, value)
            else if header == "" then
                csvLines[0] = concat(csvLines[0], value[0])
                csvLines[1] = concat(csvLines[1], value[1])
            else
                csvLines[0].push(protect(header))
                csvLines[1].push(protect(value))
        else if type == "pca" or type == "df" or k == "fonte" then
            csvLines[0].push(protect(header))
            csvLines[1].push(protect(""))

    return csvLines

convertAll(json, type)
    csvLines = []
    len = json.length

    if len > 0 then
        csvLines = convertOne(json[0], type)

        for i = 1; i < len; i++
            aux = convertOne(json[i], type)
            delete aux[0]
            csvLines = concat(csvLines, aux)

    return csvLines

json2csv(json, type)

    if type in types then
        if type[last] == "s" then
            delete type[last]

            if type == "legislacoe" then
                type = "legislacao"

            csvLines = convertAll(json, type)
        else
            csvLines = convertOne(json, type)
    else
        throw("Não é possível exportar para CSV nesta rota...")

    return joinLines(csvLines)
\end{lstlisting}

\chapter{Exemplos}

\section{Conversão de \acrshort{json} para \acrshort{xml}}\label{conv:jsonTOxml}

\subsection{\acrshort{json} a converter}\label{exem:json}
\begin{lstlisting}[language=json, caption=\acrshort{json} exemplo a converter]
{
    "nivel": 2,
    "pai": {
        "codigo": "100",
        "titulo": "ORDENAMENTO JURÍDICO E NORMATIVO"
    },
    "codigo": "100.10",
    "titulo": "Elaboração de diplomas jurídico-normativos",
    "descricao": "Compreende os processos de elaboração/alteração de legislação, de regulamentos e de diretivas políticas ou operacionais portuguesas.",
    "status": "A",
    "filhos": [],
    "notasAp": [
        {
            "idNota": "http://jcr.di.uminho.pt/m51-clav#na_c100.10_MRIKl-RBu_2sz5u9FzPqH",
            "nota": "Qualquer despacho com diretrizes gerais e abstratas"
        }
    ],
    "exemplosNotasAp": [],
    "notasEx": [
        {
            "idNota": "http://jcr.di.uminho.pt/m51-clav#ne_c100.10_bXM5qoj-hKZt6cijQktaj",
            "nota": "- Procedimentos administrativos de classificação do património cultural devem ser considerados em \"Reconhecimentos e permissões/ Classificação e declaração de interesse ou utilidade pública\" (450.20)"
        }
    ],
    "termosInd": [],
    "temSubclasses4Nivel": false,
    "temSubclasses4NivelPCA": false,
    "temSubclasses4NivelDF": false,
    "subdivisao4Nivel01Sintetiza02": true,
    "tipoProc": "",
    "procTrans": "",
    "donos": [],
    "participantes": [],
    "processosRelacionados": [],
    "legislacao": [],
    "pca": {
        "valores": "",
        "notas": "",
        "formaContagem": "",
        "subFormaContagem": "",
        "justificacao": []
    },
    "df": {
        "valor": "NE",
        "nota": null,
        "justificacao": []
    }
}
\end{lstlisting}

\subsection{\acrshort{xml} gerado}
\begin{lstlisting}[language=xml, caption=\acrshort{xml} resultante da conversão do \acrshort{json} presente em~\ref{exem:json}]
<?xml version="1.0" encoding="utf-8"?>
<root>
    <nivel type="number">2</nivel>
    <pai type="object">
        <codigo type="string">100</codigo>
        <titulo type="string">ORDENAMENTO JURÍDICO E NORMATIVO</titulo>
    </pai>
    <codigo type="string">100.10</codigo>
    <titulo type="string">Elaboração de diplomas jurídico-normativos</titulo>
    <descricao type="string">Compreende os processos de elaboração/alteração de legislação, de regulamentos e de diretivas políticas ou operacionais portuguesas.</descricao>
    <status type="string">A</status>
    <filhos type="array">
    </filhos>
    <notasAp type="array">
        <item index="0" type="object">
            <idNota type="string">http://jcr.di.uminho.pt/m51-clav#na_c100.10_MRIKl-RBu_2sz5u9FzPqH</idNota>
            <nota type="string">Qualquer despacho com diretrizes gerais e abstratas</nota>
        </item>
    </notasAp>
    <exemplosNotasAp type="array">
    </exemplosNotasAp>
    <notasEx type="array">
        <item index="0" type="object">
            <idNota type="string">http://jcr.di.uminho.pt/m51-clav#ne_c100.10_bXM5qoj-hKZt6cijQktaj</idNota>
            <nota type="string">- Procedimentos administrativos de classificação do património cultural devem ser considerados em &quot;Reconhecimentos e permissões/ Classificação e declaração de interesse ou utilidade pública&quot; (450.20)</nota>
        </item>
    </notasEx>
    <termosInd type="array">
    </termosInd>
    <temSubclasses4Nivel type="boolean">false</temSubclasses4Nivel>
    <temSubclasses4NivelPCA type="boolean">false</temSubclasses4NivelPCA>
    <temSubclasses4NivelDF type="boolean">false</temSubclasses4NivelDF>
    <subdivisao4Nivel01Sintetiza02 type="boolean">true</subdivisao4Nivel01Sintetiza02>
    <tipoProc type="string"></tipoProc>
    <procTrans type="string"></procTrans>
    <donos type="array">
    </donos>
    <participantes type="array">
    </participantes>
    <processosRelacionados type="array">
    </processosRelacionados>
    <legislacao type="array">
    </legislacao>
    <pca type="object">
        <valores type="string"></valores>
        <notas type="string"></notas>
        <formaContagem type="string"></formaContagem>
        <subFormaContagem type="string"></subFormaContagem>
        <justificacao type="array">
        </justificacao>
    </pca>
    <df type="object">
        <valor type="string">NE</valor>
        <nota type="object">
        </nota>
        <justificacao type="array">
        </justificacao>
    </df>
</root>
\end{lstlisting}

\section{Conversão de \acrshort{json} para \acrshort{csv}}\label{conv:jsonTOcsv}

\subsection{\acrshort{csv} gerado}
\begin{lstlisting}[language=pseudocode, caption=\acrshort{csv} resultante da conversão do \acrshort{json} presente em~\ref{exem:json}]
"Código";"Título";"Descrição";"Notas de aplicação";"Exemplos de NA";"Notas de exclusão";"Termos Indice";"Tipo de processo";"Processo transversal (S/N)";"Donos do processo";"Participante no processo";"Tipo de intervenção do participante";"Código do processo relacionado";"Título do processo relacionado";"Tipo de relação entre processos";"Diplomas jurídico-administrativos REF Ids";"Diplomas jurídico-administrativos REF Títulos";"Prazo de conservação administrativa";"Nota ao PCA";"Forma de contagem do PCA";"Sub Forma de contagem do PCA";"Critério PCA";"ProcRefs/LegRefs PCA";"Destino final";"Notas ao DF";"Critério DF";"ProcRefs/LegRefs DF"
"100.10";"Elaboração de diplomas jurídico-normativos";"Compreende os processos de elaboração/alteração de legislação, de regulamentos e de diretivas políticas ou operacionais portuguesas.";"Qualquer despacho com diretrizes gerais e abstratas";"";"- Procedimentos administrativos de classificação do património cultural devem ser considerados em ""Reconhecimentos e permissões/ Classificação e declaração de interesse ou utilidade pública"" (450.20)";"";"";"";"";"";"";"";"";"";"";"";"";"";"";"";"";"";"NE";"";"";""
\end{lstlisting}

\chapter{\textit{Scripts}}

\section{Instalação do \texttt{acme.sh}}\label{script:acme}
\begin{lstlisting}[language=bash, caption=\textit{Script} de instalação do \texttt{acme.sh}]
#!/bin/bash
#Used acme.sh script from https://github.com/acmesh-official/acme.sh
#necessary packages: openssl, cron, curl

###### Arguments ########
flags=1
while getopts 'c:a:' option
do
    case "${option}" in
        c) CERT_FOLDER=${OPTARG}
           ((flags += 2));;
        a) FOLDER=${OPTARG}
           ((flags += 2));;
        *) exit 1 ;;
    esac
done

CERT_FOLDER=${CERT_FOLDER:-/etc/nginx/acme.sh}
FOLDER=${FOLDER:-~/.acme.sh}

if [[ $# -ge $flags ]]; then
    DOMAINS="${@:$flags}"
else
    DOMAINS=(<put here the default domains>)
fi


NAME=acme.sh
EXEC=$FOLDER/$NAME
WEBDIR=/var/www/html

##########################

function join_by {
    local d=$1
    shift
    echo -n "$1"
    shift
    printf "%s" "${@/#/$d}"
}

#Download acme.sh script
downloadInstallScript() {
    #Install necessary packages
    local packages=('openssl' 'curl')
    local toInstall

    for p in "${packages[@]}"; do
        which $p 2> /dev/null > /dev/null
        if [ $? -ne 0 ]; then
            toInstall+=($p)
        fi
    done

    which crontab 2> /dev/null > /dev/null
    local haveCrontab=$?

    declare -A pkg_mngs
    pkg_mngs[apk]="apk --no-cache add -f"
    pkg_mngs[apt-get]="apt-get install -y"
    pkg_mngs[dnf]="dnf install -y"
    pkg_mngs[yum]="yum -y install"
    pkg_mngs[zypper]="zypper install -y"
    pkg_mngs[pacman]="pacman -S --noconfirm"

    declare -A cron
    cron[apt-get]="cron"
    cron[dnf]="crontabs"
    cron[yum]="crontabs"
    cron[zypper]="cron"
    cron[pacman]="cronie"

    local pkg_mng
    for pm in "${!pkg_mngs[@]}"; do
        if [ -x "$(command -v $pm)" ]; then
            pkg_mng="${pkg_mngs[$pm]}"

            if [[ ! -z "${cron[$pm]}" ]] && [[ $haveCrontab -ne 0 ]]; then
                toInstall+=(${cron[$pm]})
            fi
            break
        fi
    done

    if [[ ! -z $pkg_mng ]] && [[ ${#toInstall[@]} -gt 0 ]]; then
        sudo $pkg_mng ${toInstall[@]}
    fi

    touch /var/spool/cron/crontabs/root

    #Check if script not exists
    if [ ! -f "$EXEC" ]; then
        curl -O https://raw.githubusercontent.com/Neilpang/acme.sh/master/acme.sh
        chmod +x $NAME
        ./$NAME --install
        rm $NAME
        $EXEC --upgrade --auto-upgrade
    fi
}

getCertificate() {
    local domains="-d $(join_by ' -d ' ${DOMAINS[@]})"

    if [ ! -d $WEBDIR ]; then
        mkdir -p $WEBDIR
    fi

    $EXEC --issue $domains -w $WEBDIR
}

installCertificate() {
    if [ ! -d $CERT_FOLDER ]; then
        mkdir -p $CERT_FOLDER
    fi

    #gen dhparam
    openssl dhparam -out $CERT_FOLDER/dhparam.pem 2048

    $EXEC --install-cert -d $DOMAINS \
        --key-file $CERT_FOLDER/key.pem \
        --fullchain-file $CERT_FOLDER/fullchain.pem \
        --reloadcmd "nginx -s reload"
}

downloadInstallScript
getCertificate
installCertificate
\end{lstlisting}

%%\chapter{Tooling}
%Tooling: Should this be the case.

%Anyone using \Latex\ should consider having a look at \TUG,
%the \tug{\TeX\ Users Group}
