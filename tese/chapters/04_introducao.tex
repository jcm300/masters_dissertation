Vemos atualmente a mudança de paradigma em várias organizações e governos em relação a políticas e estratégias para a disponibilização de dados abertos nos domínios das ciências e da \acrfull{ap}. Quanto à \acrlong{ap} portuguesa têm sido promovidas políticas para a sua transformação digital com o objetivo de otimização de processos, a modernização de procedimentos administrativos e a redução de papel. De certa forma a agilização de procedimentos da \acrlong{ap} portuguesa.~\cite{clav}

De forma a alcançar estes objetivos a \acrlong{ap} tem desmaterializado processos e tem promovido a adoção de sistemas de gestão documental eletrónica bem como da digitalização de documentos destinados a serem arquivados.~\cite{clav}

Por forma a continuar esta transformação da \acrshort{ap} a \acrfull{dglab} apresentou a iniciativa da \acrfull{lc} para a classificação e avaliação da informação pública. A \acrshort{lc} serve de referencial para a construção normalizada dos planos de classificação e tabelas de seleção das entidades que executam funções do Estado.~\cite{clav}

Neste contexto, nasce o projeto \acrfull{clav} sendo um dos seus objetivos primordiais a operacionalização da utilização da \acrshort{lc}, numa colaboração entre a \acrshort{dglab} e a \acrfull{um} e financiado pelo SIMPLEX\footnote{Programa de Simplificação Administrativa e Legislativa}.~\cite{clav}

A plataforma \acrshort{clav} disponibiliza em formato aberto uma ontologia com as funções e processos de negócio das entidades que exercem funções públicas (ou seja a \acrshort{lc}) associadas a um catálogo de legislação e de organismos. Desta forma, a \acrshort{clav} viabiliza a desmaterialização dos procedimentos associados à elaboração de tabelas de seleção tendo como base a \acrshort{lc} e ao controlo de eliminação e arquivamento da informação pública através da integração das tabelas de seleção nos sistemas de informação das entidades públicas alertando-as quando determinado documento deve ser arquivado ou eliminado. Esta integração promove também a interoperabilidade através da utilização de uma linguagem comum (a \acrshort{lc}) usada no registo, na classificação e na avaliação da informação pública.~\cite{clav}

\section{Motivação}

A continuação do desenvolvimento da \acrshort{api} de dados da \acrshort{clav} nesta dissertação, seguindo uma metodologia \acrshort{rest}\footnote{Mais informação em \url{https://www.ics.uci.edu/~fielding/pubs/dissertation/rest_arch_style.htm}}, permite a processos ou aplicações aceder aos dados sem a intervenção humana para além de suportar a plataforma \acrshort{clav}. Um dos objetivos da \acrshort{api} de dados é permitir futuramente a criação de novas aplicações através desta. Como tal, é extremamente essencial que a \acrshort{api} de dados da \acrshort{clav} possua uma boa documentação ajudando futuros programadores ou utilizadores a utilizar a \acrshort{api}. Além disso, uma \acrshort{api} sem uma boa documentação de como a usar é inútil. Advém daí a necessidade de nesta dissertação realizar a documentação da \acrshort{api} de dados em \textit{Swagger}.

Apesar de o projeto ter em mente a disponibilização aberta de informação pública é necessário controlar a adição, edição e eliminação da informação presente na \acrlong{lc}, bem como a informação de utilizadores, da legislação, das entidades, etc, mantendo-a consistente e correta. É, portanto, necessário controlar os acessos à \acrshort{api} de dados com múltiplos níveis de acesso restringindo as operações que cada utilizador pode realizar consoante o seu nível. Desta forma garante-se que apenas pessoal autorizado pode realizar modificações aos dados.

Este controlo de acesso exige a existência de formas de autenticação. Como um cofre para o qual ninguém tem a chave não é útil pelo facto de que algo lá guardado ficará eternamente inacessível, também algo com controlo de acesso seria inútil caso não fosse possível ultrapassar esse controlo de alguma forma. Assim, uma das formas de autenticação usadas, Autenticação.gov, criada pelo Estado português, permite a autenticação dos cidadãos portugueses nos vários serviços públicos~\cite{authgov} entre os quais, a Segurança Social, o Serviço Nacional de Saúde e a Autoridade Tributária Aduaneira. Sendo este um projeto do Governo Português, a autenticação na \acrshort{clav} através do Autenticação.gov é um requisito.

Por forma a contrariar o aumento da complexidade da \acrshort{api} de dados com a adição do controlo de acesso e da autenticação pretende-se investigar se a criação de uma \textit{\acrshort{api} Gateway} simplifica a comunicação entre interface/utilizadores e a \acrshort{api} de dados.

Por fim, um dos requisitos para usar o Autenticação.gov em produção é a plataforma \acrshort{clav} estar com
\acrshort{https}. Além disso, o Centro Nacional de Cibersegurança avisou para a necessidade da plataforma \acrshort{clav}
estar com \acrshort{https} aumentando a segurança da plataforma. Assim, é necessário nesta dissertação realizar a 
migração de \acrshort{http} para \acrshort{https} da plataforma \acrshort{clav}.

\section{Objetivos}
Resumidamente, os objetivos desta dissertação são:

\begin{itemize}
    \item Documentação em \textit{Swagger} da \acrshort{api} de dados da \acrshort{clav}
    \item Adição de formatos de exportação à \acrshort{api} de dados da \acrshort{clav} (para além do já presente \acrshort{json}, adicionar \acrshort{csv}, \acrshort{xml} e \acrshort{rdf})
    \item (Continuação da) Integração do Autenticação.gov na \acrshort{clav} (adição da autenticação através de \acrlong{cmd})
    \item Proteção da \acrshort{api} de dados da \acrshort{clav} com múltiplos níveis de acesso
    \item Migração da plataforma \acrshort{clav} de \acrshort{http} para \acrshort{https}
    \item Criação de uma arquitetura aplicacional mais eficiente em termos de funcionamento e gestão (que inclui uma \textit{\acrshort{api} Gateway})
    \item Integração da \acrshort{clav} no \acrshort{iap}
\end{itemize}

\section{Questões de investigação}

Nesta dissertação, durante o seu desenvolvimento, levantaram-se algumas questões de investigação.

Sendo um dos objetivos a documentação da \acrshort{api} de dados, uma questão que surge desde logo é qual a melhor forma de documentar a \acrshort{api}.

Por outro lado, com a investigação da criação de uma \textit{\acrshort{api} Gateway} na \acrshort{clav} surgiu a
necessidade de autonomizar o serviço de autenticação (microserviço independente da \acrshort{api} de dados) por forma a
respeitar os requisitos da autenticação e proteção da \acrshort{api} de dados. Será que esta autonomização traz 
benefícios à aplicação? Além disso, com a maior modularização da \acrshort{api} de dados em microserviços será que a 
\textit{\acrshort{api} Gateway} permite melhorar esta arquitetura?

Estas questões são respondidas nesta dissertação, respondendo à primeira ao apresentar uma forma de produzir a 
documentação da \acrshort{api} de dados e respondendo à terceira com a investigação das vantagens e desvantagens de uma 
\textit{\acrshort{api} Gateway}.

Quanto aos benefícios da autonomização do serviço de autenticação passam por:
\begin{itemize}
    \item os pedidos sem autenticação não chegam agora à \acrshort{api} de dados reduzindo a carga sobre esta;
    \item o serviço de autenticação, sendo este um microserviço, pode ser colocado numa máquina à parte;
    \item simplificação do código da \acrshort{api} de dados, melhorando a facilidade de manutenção deste;
    \item maior facilidade na manutenção da autenticação da \acrshort{api} de dados, por exemplo na alteração das permissões de acesso de determinada rota.
\end{itemize}

\section{Estrutura da dissertação}

Esta dissertação começa no capítulo~\ref{cap:soa_clav} por apresentar o ponto de partida desta dissertação quanto ao estado da \acrshort{clav}. 

De seguida, abordam-se os vários objetivos da dissertação, um por capítulo, a proteção da \acrshort{api} de dados no capítulo~\ref{cap:prot_api}, a adição da autenticação através de \acrshort{cmd} no capítulo~\ref{cap:cmd}, a documentação da \acrshort{api} da \acrshort{clav} no capítulo~\ref{cap:doc}, a adição de formatos exportação à \acrshort{api} de dados no capítulo~\ref{cap:exp}, a migração de \acrshort{http} para \acrshort{https} no capítulo~\ref{cap:https} e a criação de uma \textit{\acrshort{api} Gateway} no capítulo~\ref{cap:api_gateway}. Em cada um destes capítulos inicia-se com a apresentação do estado da arte e consequente solução e implementação da solução por forma a atingir o objetivo.

Após a apresentação destes capítulos no capítulo~\ref{cap:deploy} apresenta-se de que forma pode ser feita a instalação das duas versões desenvolvidas durante esta tese, a versão sem a presença do \textit{Kong} e a versão com \textit{Kong}. É, além disso, realizada uma comparação das duas versões.

Finalmente, no capítulo~\ref{cap:conc} são apresentadas as conclusões finais e os trabalhos futuros a efetuar.