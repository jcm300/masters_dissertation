%TODO
TODO

\section{Trabalho Futuro}

Foram alcançados todos os objetivos desta dissertação, contudo certas tarefas requerem uma manutenção constante. Uma delas é a documentação em \textit{OpenAPI} que necessita de ser atualizada sempre que são adicionadas novas rotas à \acrshort{api} ou quando alguma das rotas presentes mude, seja na resposta, nos parâmetros, na funcionalidade ou ainda caso esta passe a \textit{deprecated}. Outra tarefa que requer manutenção é a exportação para \acrshort{csv}. Se se pretender que mais alguma rota possua exportação para \acrshort{csv} ou caso alguma das respostas das rotas que possuem exportação para \acrshort{csv} sejam alteradas é necessário adicionar ou alterar, respetivamente, no conversor de \acrshort{json} para \acrshort{csv} como deve ser feita a transformação dos objetos \acrshort{json} de saída da rota. Há ainda outra tarefa de manutenção e esta deve também ser realizada sempre que é adicionada uma nova rota. Esta tarefa consiste em adicionar ao serviço de autenticação quem pode aceder a essa rota, ao adicionar as permissões dessa rota no dicionário de permissões das rotas presente nesse serviço.

Já quanto à continuação do trabalho desenvolvido por esta dissertação destaca-se a \acrshort{api} de dados com \textit{Kong}. Apesar do que foi desenvolvido estar num estado passível de ser usado em produção, há varias melhorias que podem ser desenvolvidas desde replicar a \acrshort{api} de dados colocando o \textit{Kong} a servir de \textit{load balacing} até colocar \textit{rate limiting} especificamente para cada rota com o objetivo principal de melhorar a performance e a tolerância a falhas da \acrshort{api} de dados.
