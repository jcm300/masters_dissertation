\section{JavaScript e a Programação Assíncrona}
A linguagem \textit{JavaScript} teve a sua origem no \textit{browser} \textit{Netscape Navigator} como uma forma de adicionar programas a páginas web.\cite{elojs} Hoje em dia é usado por grande parte dos \textit{browsers} e tornou possível as aplicações web onde o utilizador pode interagir sem precisar de realizar \textit{refresh} ao fim de cada ação.
Contudo o \textit{Javascript} é bastante liberal no que permite o programador escrever, facilitando a aprendizagem para novos programadores mas tornando a tarefa de resolver e encontrar problemas bem mais difícil visto não apontar aonde estão esses problemas. Ainda assim esta flexibilidade permite uma quantidade de técnicas que não são possíveis em linguagens mais rígidas e que permitem ultrapassar algumas falhas do \textit{Javascript}.\cite{elojs} 
%javascript how manage the processes and awaits etc
\cite{elojs}

\subsection{Callbacks}

\subsection{Promessas}

\section{REST}
\cite{restws}

\section{express}
\cite{wdmongo}
%procurar "semelhantes" para cada um

\section{passport}
%procurar "semelhantes" para cada um

\section{jsonwebtoken}
%package e conceito JWT
%procurar "semelhantes" para cada um

\section{passport-jwt}
%procurar "semelhantes" para cada um

\section{CORS}
%falar da package e do conceito CORS
%procurar "semelhantes" para cada um

\section{axios}
%procurar "semelhantes" para cada um

\section{HTTP Status}

\section{Headers do HTTP}

\section{Autenticação.gov}
\cite{agov}

\section{exceljs}
%procurar "semelhantes" para cada um

\section{MongoDB}
\cite{wdmongo}

\section{mongoose}
%procurar "semelhantes" para cada um

\section{Web Semântica}
\cite{lsparql}

\subsection{RDF}
\cite{lsparql}

\subsection{SPARQL}
\cite{lsparql}

%se calhar referir algumas packages de ligação ao sparql em vez de usar o que criei

\section{GraphDB}
%procurar "semelhantes" para cada um (Neo4j)

\section{Swagger}
%procurar "semelhantes" para cada um

\section{Swagger-UI}

\section{yaml-include}
%procurar "semelhantes" para cada um

\section{swagger-ui-express}
%procurar "semelhantes" para cada um

\section{js-yaml}
%procurar "semelhantes" para cada um

\section{Nginx}
\cite{nginxcook}
%procurar "semelhantes" para cada um

\section{Ontologia}
\cite{bontology}
%conceito

\section{Docker}
\cite{udocker}
%procurar "semelhantes" para cada um

\section{Docker Compose}
\cite{udocker}
%procurar "semelhantes" para cada um

\section{Lista Consolidada}

\section{Tabelas de Seleção}

\section{Cache e Fecho Transitivo}
